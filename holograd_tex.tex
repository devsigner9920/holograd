% main.tex (fully consistent version)
% Key fixes:
%  - \v -> \vdir (avoid LaTeX accent collision)
%  - a^\* -> a^{*} (safe)
%  - Dir(s) is FIXED: unit-norm isotropic on sphere => E[vv^T] = (1/D) I
%  - sigma^2 is defined and used consistently in BOTH text + algorithm
%  - ADC: FIXED z distribution (no normalization): z ~ N(0, I_r) so E[zz^T]=I_r
%  - Subspace case uses v = U z / ||Uz|| (optional) OR v = U z with scale-corrected sigma^2
%    In this paper we pick the clean, consistent choice: v = U z (no normalization),
%    which yields E[vv^T] = U U^T (i.e., sigma^2 = 1 in-subspace).
%
% Compile:
%   pdflatex main
%   bibtex main
%   pdflatex main
%   pdflatex main

\documentclass[11pt]{article}

\usepackage[margin=1in]{geometry}
\usepackage{microtype}
\usepackage{times}
\usepackage{amsmath, amssymb, amsthm}
\usepackage{mathtools}
\usepackage{bm}
\usepackage{graphicx}
\usepackage{booktabs}
\usepackage{enumitem}
\usepackage{hyperref}
\usepackage{url}

\usepackage{algorithm}
\usepackage{algpseudocode}

% -------------------------
% Theorem environments
% -------------------------
\newtheorem{theorem}{Theorem}
\newtheorem{lemma}{Lemma}
\newtheorem{definition}{Definition}
\newtheorem{assumption}{Assumption}
\newtheorem{proposition}{Proposition}
\newtheorem{corollary}{Corollary}

% -------------------------
% Notation macros (SAFE)
% -------------------------
\newcommand{\R}{\mathbb{R}}
\newcommand{\E}{\mathbb{E}}
\newcommand{\Var}{\mathrm{Var}}
\newcommand{\ip}[2]{\left\langle #1, #2 \right\rangle}
\newcommand{\norm}[1]{\left\lVert #1 \right\rVert}
\newcommand{\Lcal}{\mathcal{L}}
\newcommand{\g}{\bm{g}}
\newcommand{\vdir}{\bm{v}}   % renamed from \v (reserved accent in LaTeX)
\newcommand{\U}{\bm{U}}
\newcommand{\thetaVec}{\bm{\theta}}
\newcommand{\DeltaTheta}{\Delta \bm{\theta}}

% Identifiers that contain underscores (safe in text mode)
\newcommand{\codebookid}{\texttt{codebook\_id}}
\newcommand{\batchid}{\texttt{batch\_id}}
\newcommand{\pverifyid}{\texttt{p\_verify}}
\newcommand{\tqrid}{\texttt{T\_qr}}
\newcommand{\keffid}{\texttt{K\_eff}}

% -------------------------
% Title
% -------------------------
\title{\textbf{HoloGrad: Proof-of-Gradient-Projection (PoGP) for Verifiable, Mining-Style Distributed LLM Pretraining}}
\author{
  WonJune Kang\\
  \texttt{devsigner9920@gmail.com}
}
\date{}

\begin{document}
\maketitle

\begin{abstract}
Large-scale language model (LLM) pretraining is bottlenecked by high-dimensional gradient/parameter tensor synchronization and straggler effects, effectively restricting training to data-center-grade clusters. We propose \emph{HoloGrad}, a protocol and system that replaces tensor communication with \emph{verifiable scalar proofs} based on gradient projections (directional derivatives). Each worker submits only a seed and a scalar projection value, while the coordinator reconstructs directions from seeds and aggregates updates via linear synthesis. We introduce \emph{Proof-of-Gradient-Projection (PoGP)}: a sampling-based verification mechanism enabling low-cost validation and incentive-compatible accounting in a mining-style open network.

We identify a fundamental challenge: for neural networks, gradient directions vary significantly across mini-batches (pairwise cosine similarity $\approx 0.07$), preventing random-direction estimators from achieving efficient reconstruction when $K \ll D$. To address this, we propose two complementary variants: (i) \emph{HoloGrad-ADC} with an Adaptive Direction Codebook for settings where gradients occupy a stable low-rank subspace, and (ii) \emph{Momentum-Centric HoloGrad}, which projects onto a shared momentum direction, reducing communication to a single scalar per worker while achieving meaningful convergence ($\approx 30$--$40\%$ of full SGD efficiency). We provide descent guarantees, analyze verification/slashing parameters, and present reproducible experiments demonstrating when each variant is appropriate.
\end{abstract}

\section{Introduction}
\paragraph{Background.}
Distributed training typically requires communicating high-dimensional gradients (or equivalent sufficient statistics), inducing bandwidth and latency bottlenecks. Synchronous training further suffers from stragglers, making permissionless heterogeneous nodes impractical.

\paragraph{Goal.}
We aim to reshape pretraining into a mining-like protocol: permissionless participation, cheap verification, and incentive-compatible rewards—while still improving a global model.

\paragraph{Key idea.}
Replace tensor communication with \emph{scalar proofs}:
\[
a = \ip{\nabla_{\thetaVec}\Lcal_B(\thetaVec)}{\vdir}, \quad a\in\R
\]
where $\vdir$ is deterministically generated from a public seed. The coordinator reconstructs $\vdir$ and synthesizes updates from scalar-direction pairs. Directional-derivative-based gradient estimators are connected to forward-mode AD and "gradients without backpropagation" directions \cite{baydin2022gradients,shukla2023randomized}.

\paragraph{Neural network challenge.}
While random projection estimators are unbiased in expectation, they face a fundamental obstacle in neural network training: \emph{gradient directions vary dramatically across mini-batches}. Empirically, we observe pairwise cosine similarities of only $\approx 0.07$ between gradients from different batches. This near-orthogonality implies that no fixed low-rank subspace captures most gradient energy across batches, causing both random-direction estimators (when $K \ll D$) and learned codebooks (ADC) to struggle. In contrast, training momentum---which accumulates past gradient information---provides a stable, informative direction that captures meaningful signal.

\paragraph{Contributions.}
We present two complementary HoloGrad variants:
\begin{itemize}[leftmargin=*]
  \item \textbf{HoloGrad-ADC}: Uses an Adaptive Direction Codebook to project onto a learned low-rank subspace. Effective when gradients occupy a stable subspace (e.g., convex problems, later training stages).
  \item \textbf{Momentum-Centric HoloGrad}: Projects each worker's gradient onto a shared momentum direction, reducing communication to a \emph{single scalar per worker}. Achieves $\approx 30$--$40\%$ of full SGD efficiency while being $64\times$ more communication-efficient than random projections.
\end{itemize}

\section{Related Work}
Directional-derivative / forward-mode gradient construction has been studied as an alternative to backpropagation \cite{baydin2022gradients,shukla2023randomized}. Communication-efficient distributed optimization includes sign-based methods such as signSGD \cite{bernstein2018signsgd}. Locking-free training via synthetic gradients decouples modules \cite{jaderberg2017dni}. Our contribution differs by \emph{protocolizing} directional-derivative scalars into verifiable proofs with economic enforcement and by specifying a fixed adaptive direction codebook update rule (Streaming Oja-QR) for sample-efficiency.

\section{Problem Formulation and Notation}
Let $\thetaVec\in\R^D$ be model parameters, $\Lcal_B(\thetaVec)$ the minibatch loss, and $\Lcal(\thetaVec)=\E_B[\Lcal_B(\thetaVec)]$ the population objective. Let $\g_B(\thetaVec)=\nabla_{\thetaVec}\Lcal_B(\thetaVec)$ and $\g(\thetaVec)=\nabla_{\thetaVec}\Lcal(\thetaVec)$.

\paragraph{System constraints.}
(i) per-worker uplink must be low-dimensional, ideally $O(1)$ scalars per task; (ii) verification must be feasible at scale via sampling; (iii) robustness to heterogeneity and adversarial behavior.

\section{HoloGrad and PoGP Protocol}

\subsection{Seeded direction generation (fixed)}
A public seed $s$ defines a direction via a deterministic generator
\[
\vdir = \mathrm{Dir}(s)\in\R^D.
\]
\textbf{We fix $\mathrm{Dir}$ to output a unit-norm isotropic direction} (e.g., sample $z\sim \mathcal{N}(0,I_D)$ from a seeded PRNG and set $\vdir=z/\norm{z}$). Then $\norm{\vdir}=1$ and
\[
\E[\vdir\vdir^\top] = \frac{1}{D}I,
\]
so throughout the full-space case we use the scale parameter
\[
\sigma^2 := \frac{1}{D}.
\]
(Other choices such as unnormalized Rademacher directions are possible but are not used in this paper to keep norms bounded.)

\subsection{Worker proof: scalar projection}
Given current checkpoint $\thetaVec_t$ and minibatch $B_t$, a worker computes
\[
a_{t} = \ip{\g_{B_t}(\thetaVec_t)}{\vdir}
\]
and submits a proof $(s,a_t)$ plus metadata (optional attestation, timing, signature).

\subsection{Coordinator aggregation (HoloGrad synthesis)}
Collect $K$ proofs $(s_j,a_j)$, reconstruct $\vdir_j=\mathrm{Dir}(s_j)$, and synthesize the gradient estimate using the scale correction:
\[
\widehat{\g}_t = \frac{1}{K}\sum_{j=1}^K \frac{1}{\sigma^2}\,a_{t,j}\vdir_{t,j},
\qquad
\thetaVec_{t+1} = \thetaVec_t - \eta \widehat{\g}_t.
\]
With the fixed unit-norm isotropic $\mathrm{Dir}$ above, $\sigma^2=1/D$ and thus $\widehat{\g}_t=\frac{D}{K}\sum_j a_{t,j}\vdir_j$.

\subsection{PoGP verification}
A verifier samples each proof with probability $p_{\mathrm{verify}}$ and recomputes $a^{*}$ under the same $(\thetaVec_t,B_t,s)$:
\[
|a-a^{*}|\le \epsilon \Rightarrow \text{accept}; \quad \text{else slash/penalize}.
\]
This enables low-cost statistical verification in expectation via sampling; $\epsilon$ handles numeric nondeterminism and implementation differences.

\section{Theory: Correctness and Descent}

\begin{assumption}[Isotropic directions up to scale]\label{assump:isotropic}
Directions satisfy $\E[\vdir\vdir^\top]=\sigma^2 I$ for some $\sigma^2>0$.
\end{assumption}

\begin{theorem}[Unbiased directional estimator with scale correction]\label{thm:unbiased}
Let $\hat{\g}(\vdir) = \frac{1}{\sigma^2}(\g\cdot \vdir)\vdir$. Under Assumption~\ref{assump:isotropic},
\[
\E_{\vdir}[\hat{\g}(\vdir)] = \g.
\]
\end{theorem}
\begin{proof}
\[
\E\!\left[\frac{1}{\sigma^2}(\g\cdot \vdir)\vdir\right]
= \frac{1}{\sigma^2}\E[\vdir\vdir^\top]\g
= \frac{1}{\sigma^2}(\sigma^2 I)\g
= \g.
\]
\end{proof}

\paragraph{Variance and sample size.}
With i.i.d. directions, $\widehat{\g} = \frac{1}{K}\sum_{j=1}^K \hat{\g}(\vdir_j)$ has variance scaling $O(1/K)$; heavy tails increase the constant factors \cite{shukla2023randomized}.

\subsection{Adaptive Direction Codebook (ADC)}
Random directions can be sample-inefficient in high dimensions. We restrict directions to a learned low-rank subspace.

\paragraph{Fixed $z$ distribution (no normalization).}
We \textbf{fix} $z\sim\mathcal{N}(0,I_r)$ (or Rademacher $\pm1$) without normalization so that
\[
\E[zz^\top]=I_r.
\]
This avoids the ``$1/r$'' scaling ambiguity.

\paragraph{Codebook and subspace directions.}
Maintain an orthonormal codebook $\U_t\in\R^{D\times r}$ with $r\ll D$. A seed generates $z\in\R^r$ and
\[
\vdir = \U_t z.
\]
Then $\E[\vdir\vdir^\top] = \U_t\E[zz^\top]\U_t^\top = \U_t\U_t^\top$. In the subspace case we interpret Assumption~\ref{assump:isotropic} in the projected form
\[
\E[\vdir\vdir^\top] = \sigma^2 \U\U^\top,
\quad\text{with }\sigma^2=1 \text{ under } \E[zz^\top]=I_r.
\]
Accordingly, the scale-corrected estimator $\hat{\g}(\vdir)=\frac{1}{\sigma^2}(g\cdot \vdir)\vdir$ satisfies
\[
\E[\hat{\g}(\vdir)] = \U\U^\top \g,
\]
i.e., it is unbiased for the projected gradient.

\begin{assumption}[$L$-smoothness]\label{assump:smooth}
$\Lcal$ is $L$-smooth: for any $\delta$,
\[
\Lcal(\thetaVec+\delta)\le \Lcal(\thetaVec)+\ip{\nabla\Lcal(\thetaVec)}{\delta}+\frac{L}{2}\norm{\delta}^2.
\]
\end{assumption}

\begin{theorem}[Projected-gradient descent decrease (idealized)]\label{thm:subspace}
Under Assumption~\ref{assump:smooth}, with deterministic update $\delta=-\eta \U\U^\top \g$,
\[
\Lcal(\thetaVec-\eta \U\U^\top \g)\le \Lcal(\thetaVec) - \Big(\eta-\frac{L\eta^2}{2}\Big)\norm{\U\U^\top \g}^2.
\]
Thus for $\eta\in(0,2/L)$, the loss decreases whenever $\U\U^\top \g\neq 0$.
\end{theorem}
\begin{proof}
Apply $L$-smoothness with $\delta=-\eta \U\U^\top \g$. The linear term is $-\eta\norm{\U\U^\top \g}^2$ and the quadratic term is $\frac{L\eta^2}{2}\norm{\U\U^\top \g}^2$.
\end{proof}

\paragraph{Remark (stochasticity and robust aggregation).}
In the full protocol, $\DeltaTheta_t$ is stochastic (mini-batches, random directions) and may be biased by robust aggregation (e.g., trimming) under adversaries. A full convergence theorem requires bounding both the variance term and any aggregation-induced bias; we treat this as an experimental and future-theoretical extension.

\section{Fixed Codebook Update: Streaming Oja-QR}
We now \emph{fix} the codebook update rule (no ambiguity).

\paragraph{Observation.}
The synthesized update provides a streaming signal of frequent update directions. We track a top-$r$ subspace via Oja's rule with periodic orthonormalization.

\subsection{Streaming Oja step}
Given the aggregated gradient estimate $\widehat{\g}_t$ (or parameter update direction proportional to it) and current $\U_t$:
\[
\widetilde{\U}_{t+1} = \U_t + \alpha_t\, \widehat{\g}_t(\widehat{\g}_t^\top \U_t).
\]
Periodically (every $T_{\mathrm{qr}}$ steps), perform QR:
\[
\U_{t+1} \leftarrow \mathrm{QR}(\widetilde{\U}_{t+1}).
\]
Otherwise set $\U_{t+1}\leftarrow \widetilde{\U}_{t+1}$ with column normalization.

\paragraph{Captured energy ratio.}
Define
\[
\gamma_t=\frac{\norm{\U_t\U_t^\top \g(\thetaVec_t)}^2}{\norm{\g(\thetaVec_t)}^2}\in[0,1].
\]
Then Theorem~\ref{thm:subspace} implies idealized descent is proportional to $\gamma_t\norm{\g}^2$.

\section{Gradient Subspace Variability in Neural Networks}
\label{sec:variability}

The theoretical guarantees of HoloGrad assume that either (i) enough random directions are sampled ($K$ sufficiently large), or (ii) the ADC subspace captures significant gradient energy ($\gamma_t$ close to 1). We now investigate why these assumptions often fail for neural network training.

\subsection{Empirical observation: near-orthogonal gradients}

We computed gradients on a GPT-2-small model across 50 independent mini-batches and measured pairwise cosine similarities. The results are striking:
\begin{itemize}[leftmargin=*]
  \item \textbf{Pairwise cosine similarity}: $\approx 0.07$ (nearly orthogonal)
  \item \textbf{Leave-one-out SVD reconstruction}: $\approx 0.24$ cosine similarity
  \item \textbf{Gradient variance ratio}: $\frac{\|\E[\g_B] - \g_B\|^2}{\|\g_B\|^2} \approx 0.93$
\end{itemize}

This implies that gradients from different mini-batches point in nearly independent directions in the $D$-dimensional parameter space.

\subsection{Implications for random projections}

For random unit directions $\vdir_j$ with $K \ll D$, the reconstructed gradient is:
\[
\widehat{\g} = \frac{D}{K}\sum_{j=1}^K \ip{\g}{\vdir_j}\vdir_j.
\]
The expected cosine similarity between $\widehat{\g}$ and $\g$ scales as:
\[
\E\left[\frac{\ip{\g}{\widehat{\g}}}{\norm{\g}\norm{\widehat{\g}}}\right] \approx \sqrt{\frac{K}{D}}.
\]
For GPT-2-small ($D \approx 124\text{M}$) with $K=128$, this gives $\sqrt{128/124\text{M}} \approx 0.001$---essentially noise.

\subsection{Implications for ADC}

The ADC learns a subspace from historical gradient estimates. However, if gradients are nearly orthogonal across batches, no stable low-rank subspace exists. We observed:
\begin{itemize}[leftmargin=*]
  \item Captured energy $\gamma_t < 0.01$ even with rank $r=64$
  \item Codebook directions become stale within a few batches
  \item Improvement over random directions is marginal ($< 2\times$)
\end{itemize}

\subsection{Why momentum works}

In contrast to random or learned directions, the training momentum:
\[
\bm{m}_t = \beta \bm{m}_{t-1} + (1-\beta)\g_t
\]
accumulates gradient information over time. Even if individual gradients are nearly orthogonal, the momentum direction correlates with the \emph{average} gradient direction. Empirically:
\begin{itemize}[leftmargin=*]
  \item Cosine similarity $\ip{\g_t}{\bm{m}_t/\norm{\bm{m}_t}} \approx 0.18$
  \item This is $\approx 180\times$ better alignment than random directions
  \item Communication: 1 scalar vs $K=128$ scalars
\end{itemize}

This motivates our Momentum-Centric HoloGrad variant.

\section{Momentum-Centric HoloGrad}
\label{sec:momentum}

We now present a variant of HoloGrad that uses the coordinator's momentum as the shared direction, achieving extreme communication efficiency.

\subsection{Protocol overview}

Instead of sampling $K$ random directions, we use a single shared direction: the normalized momentum maintained by the coordinator.

\paragraph{Coordinator state.}
Maintain momentum $\bm{m}_t \in \R^D$ and checkpoint $\thetaVec_t$.

\paragraph{Worker task.}
Given $(\thetaVec_t, B_t, \bm{m}_t)$, compute:
\[
a = \ip{\g_{B_t}(\thetaVec_t)}{\frac{\bm{m}_t}{\norm{\bm{m}_t}}}.
\]
Submit scalar $a$ (one number per worker).

\paragraph{Coordinator aggregation.}
Collect scalars $\{a_i\}_{i=1}^N$ from $N$ workers. Compute aggregated projection:
\[
\bar{a} = \frac{1}{N}\sum_{i=1}^N a_i.
\]
Estimate gradient magnitude from moving average of $\|\widehat{\g}\|$, denoted $\hat{\sigma}$. Reconstruct:
\[
\widehat{\g}_t = \bar{a} \cdot \hat{\sigma} \cdot \frac{\bm{m}_t}{\norm{\bm{m}_t}}.
\]
Update parameters and momentum:
\begin{align}
\thetaVec_{t+1} &= \thetaVec_t - \eta \widehat{\g}_t, \\
\bm{m}_{t+1} &= \beta \bm{m}_t + (1-\beta)\widehat{\g}_t.
\end{align}

\subsection{Theoretical guarantees}

We now establish formal guarantees for momentum-centric HoloGrad. Let $\hat{\bm{m}}_t = \bm{m}_t / \norm{\bm{m}_t}$ denote the unit momentum direction.

\begin{assumption}[Unbiased stochastic gradients]\label{assump:unbiased}
Each worker's mini-batch gradient $\g_i$ satisfies $\E[\g_i] = \g$ where $\g = \nabla\Lcal(\thetaVec_t)$ is the true gradient.
\end{assumption}

\begin{theorem}[Non-negative expected alignment]\label{thm:alignment}
Under Assumption~\ref{assump:unbiased}, the reconstructed gradient $\widehat{\g}_t = \bar{a} \cdot \hat{\bm{m}}_t$ satisfies
\[
\E[\ip{\g}{\widehat{\g}_t}] = \ip{\g}{\hat{\bm{m}}_t}^2 \geq 0.
\]
The expected update direction always has non-negative alignment with the true gradient.
\end{theorem}

\begin{proof}
Each worker computes $a_i = \ip{\g_i}{\hat{\bm{m}}_t}$. Taking expectation:
\[
\E[\bar{a}] = \E\left[\frac{1}{N}\sum_{i=1}^N \ip{\g_i}{\hat{\bm{m}}_t}\right] = \ip{\E[\g_i]}{\hat{\bm{m}}_t} = \ip{\g}{\hat{\bm{m}}_t}.
\]
Define $\alpha_t := \ip{\g}{\hat{\bm{m}}_t}$ (the gradient's projection onto momentum). Then:
\[
\E[\widehat{\g}_t] = \E[\bar{a}] \cdot \hat{\bm{m}}_t = \alpha_t \cdot \hat{\bm{m}}_t.
\]
The expected alignment is:
\[
\E[\ip{\g}{\widehat{\g}_t}] = \ip{\g}{\alpha_t \cdot \hat{\bm{m}}_t} = \alpha_t \ip{\g}{\hat{\bm{m}}_t} = \alpha_t^2 \geq 0.
\]
\end{proof}

\begin{theorem}[Expected descent under smoothness]\label{thm:momentum-descent}
Under Assumptions~\ref{assump:smooth} and~\ref{assump:unbiased}, with update $\thetaVec_{t+1} = \thetaVec_t - \eta \widehat{\g}_t$ where $\widehat{\g}_t = \bar{a} \cdot \hat{\bm{m}}_t$:
\[
\E[\Lcal(\thetaVec_{t+1})] \leq \Lcal(\thetaVec_t) - \alpha_t^2 \left(\eta - \frac{L\eta^2}{2}\right) + \frac{L\eta^2}{2}\Var(\bar{a})
\]
where $\alpha_t = \ip{\g}{\hat{\bm{m}}_t}$ and $\Var(\bar{a}) = \frac{1}{N}\Var(a_i)$ is the variance of the aggregated scalar.
\end{theorem}

\begin{proof}
By $L$-smoothness (Assumption~\ref{assump:smooth}):
\[
\Lcal(\thetaVec_{t+1}) \leq \Lcal(\thetaVec_t) - \eta\ip{\g}{\widehat{\g}_t} + \frac{L\eta^2}{2}\norm{\widehat{\g}_t}^2.
\]
Since $\widehat{\g}_t = \bar{a} \cdot \hat{\bm{m}}_t$ and $\norm{\hat{\bm{m}}_t} = 1$:
\[
\ip{\g}{\widehat{\g}_t} = \bar{a} \cdot \alpha_t, \quad \norm{\widehat{\g}_t}^2 = \bar{a}^2.
\]
Taking expectation:
\begin{align}
\E[\Lcal(\thetaVec_{t+1})] &\leq \Lcal(\thetaVec_t) - \eta \E[\bar{a}] \alpha_t + \frac{L\eta^2}{2}\E[\bar{a}^2] \\
&= \Lcal(\thetaVec_t) - \eta \alpha_t^2 + \frac{L\eta^2}{2}\left(\alpha_t^2 + \Var(\bar{a})\right) \\
&= \Lcal(\thetaVec_t) - \alpha_t^2\left(\eta - \frac{L\eta^2}{2}\right) + \frac{L\eta^2}{2}\Var(\bar{a}).
\end{align}
\end{proof}

\begin{corollary}[Sufficient condition for descent]\label{cor:descent}
If $\eta < 2/L$ and $\alpha_t \neq 0$, then the expected loss decreases whenever
\[
\alpha_t^2 > \frac{L\eta}{2 - L\eta} \Var(\bar{a}).
\]
With $N$ workers and bounded per-worker variance $\Var(a_i) \leq \sigma_a^2$:
\[
\Var(\bar{a}) = \frac{\sigma_a^2}{N},
\]
so more workers reduce variance and ensure descent.
\end{corollary}

\paragraph{Interpretation.}
The descent guarantee depends on $\alpha_t^2 = \ip{\g}{\hat{\bm{m}}_t}^2$, the squared correlation between the true gradient and momentum. This quantity is \emph{large} when momentum tracks the gradient direction---which occurs naturally since momentum is an exponential moving average of past gradients. Empirically, we observe $|\alpha_t| / \norm{\g} \approx 0.18$, yielding meaningful descent despite projecting onto a single direction.

\paragraph{Comparison to random projections.}
For random unit directions $\vdir_j$, the expected squared alignment is $\E[\ip{\g}{\vdir_j}^2] = \norm{\g}^2 / D$. With $K$ directions:
\[
\text{Random: } \alpha_{\text{rand}}^2 \approx K \cdot \norm{\g}^2 / D.
\]
For momentum, $\alpha_t^2 \approx 0.03 \norm{\g}^2$ (based on empirical cosine similarity of 0.18). The momentum approach is more efficient than random projections when:
\[
0.03 \norm{\g}^2 > K \cdot \norm{\g}^2 / D \quad \Rightarrow \quad K < 0.03 D.
\]
For GPT-2-small ($D \approx 124$M), this means momentum outperforms random projections for any $K < 3.7$M---far beyond practical $K$ values.

\subsection{Trade-offs}

\begin{itemize}[leftmargin=*]
  \item \textbf{Communication}: $O(1)$ scalar per worker (vs $O(K)$ for random HoloGrad)
  \item \textbf{Convergence speed}: $\approx 30$--$40\%$ of full SGD efficiency
  \item \textbf{Bias}: Updates are biased toward the momentum direction
  \item \textbf{Cold start}: Requires warm-up (first few steps use random direction)
\end{itemize}

\subsection{When to use each variant}

\begin{center}
\begin{tabular}{lcc}
\toprule
\textbf{Setting} & \textbf{HoloGrad-ADC} & \textbf{Momentum-Centric} \\
\midrule
Convex problems & \checkmark & \\
Stable gradient subspace & \checkmark & \\
Neural network training & & \checkmark \\
Extreme communication limits & & \checkmark \\
Later training (fine-tuning) & \checkmark & \checkmark \\
\bottomrule
\end{tabular}
\end{center}

\section{Robust Aggregation of Scalar Proofs}
Since proofs are scalars, we can robustly aggregate using trimmed mean or median-of-means.

\paragraph{Trimmed mean (fixed default).}
Sort $\{a_j\}$ and remove top/bottom $\tau K$ values:
\[
\tilde{a}_j \leftarrow \text{TrimMean}_\tau(\{a_j\}_{j=1}^K).
\]
Then synthesize using $\sum \tilde{a}_j \vdir_j$ with the same scale correction as in Section~4.

\section{Verification and Incentives (Mining-Style Economics)}
\subsection{Sampling verification and detection probability}
Let each submitted proof be verified independently with probability $p$. Suppose an attacker submits $m$ proofs per step and a fraction $q$ are invalid (would fail verification). The probability of catching at least one invalid proof in a step is
\[
P(\mathrm{caught}) = 1 - (1-p)^{qm}.
\]

\subsection{Slashing model and incentive-compatibility}
We state an explicit \emph{step-level} slashing model:
(i) if no invalid proof is caught in a step, the attacker receives reward $R$ for each accepted invalid proof; (ii) if any invalid proof is caught, the step reward is forfeited and an additional slashing penalty $S$ is applied.
Then the expected cheating utility is
\[
\E[U_{\text{cheat}}] = (1-P(\mathrm{caught}))\cdot(qmR) - P(\mathrm{caught})\cdot S.
\]
A sufficient condition for cheating to be non-profitable is $\E[U_{\text{cheat}}]\le 0$, i.e.,
\[
S \ge \frac{1-P(\mathrm{caught})}{P(\mathrm{caught})}\,qmR
\quad\text{with}\quad
P(\mathrm{caught})=1-(1-p)^{qm}.
\]
Replication (assigning identical tasks to multiple workers) increases detection probability and further discourages attacks.

\section{Algorithms}

\begin{algorithm}[t]
\caption{HoloGrad Coordinator Step (scale-consistent)}
\label{alg:coord}
\begin{algorithmic}[1]
\Require checkpoint $\thetaVec_t$, codebook $\U_t$, minibatch ID $B_t$, number of proofs $K$, step size $\eta$, scale $\sigma^2$
\State Publish tasks: seeds $\{s_j\}_{j=1}^K$, with $(H(\thetaVec_t),H(B_t),\codebookid)$
\State Collect proofs $\{(s_j,a_j)\}_{j=1}^K$
\State Robust scalar aggregation $\{\tilde{a}_j\}\leftarrow \mathrm{TrimMean}_\tau(\{a_j\})$
\For{$j=1..K$}
  \State $\vdir_j \leftarrow \mathrm{Dir}(s_j)$ \Comment{full-space} \ \ \textbf{or} \ \ $\vdir_j \leftarrow \U_t z(s_j)$ \Comment{subspace}
\EndFor
\State $\widehat{\g}_t \leftarrow \frac{1}{K}\sum_{j=1}^K \frac{1}{\sigma^2}\,\tilde{a}_j \vdir_j$ \Comment{scale-corrected}
\State $\thetaVec_{t+1} \leftarrow \thetaVec_t - \eta \widehat{\g}_t$
\State Update codebook via Streaming Oja-QR using $\widehat{\g}_t$ (periodic QR)
\end{algorithmic}
\end{algorithm}

\begin{algorithm}[t]
\caption{Worker Proof Generation}
\label{alg:worker}
\begin{algorithmic}[1]
\Require task $(\thetaVec_t,B_t,\U_t,s)$
\State $\vdir \leftarrow \mathrm{Dir}(s)$ \Comment{full-space} \ \ \textbf{or} \ \ $\vdir \leftarrow \U_t z(s)$ \Comment{subspace}
\State Compute $a \leftarrow \ip{\nabla_{\thetaVec}\Lcal_{B_t}(\thetaVec_t)}{\vdir}$ \Comment{directional derivative}
\State Submit proof $(s,a)$ with optional signature/attestation
\end{algorithmic}
\end{algorithm}

\begin{algorithm}[t]
\caption{Momentum-Centric HoloGrad Coordinator Step}
\label{alg:momentum-coord}
\begin{algorithmic}[1]
\Require checkpoint $\thetaVec_t$, momentum $\bm{m}_t$, minibatch ID $B_t$, $N$ workers, step size $\eta$, momentum coefficient $\beta$, gradient norm estimate $\hat{\sigma}$
\State Publish task: $(\thetaVec_t, B_t, \bm{m}_t)$
\State Collect scalars $\{a_i\}_{i=1}^N$ from workers
\State $\bar{a} \leftarrow \mathrm{TrimMean}_\tau(\{a_i\}_{i=1}^N)$ \Comment{robust aggregation}
\State $\widehat{\g}_t \leftarrow \bar{a} \cdot \hat{\sigma} \cdot \frac{\bm{m}_t}{\norm{\bm{m}_t}}$ \Comment{reconstruct gradient estimate}
\State $\thetaVec_{t+1} \leftarrow \thetaVec_t - \eta \widehat{\g}_t$ \Comment{parameter update}
\State $\bm{m}_{t+1} \leftarrow \beta \bm{m}_t + (1-\beta)\widehat{\g}_t$ \Comment{momentum update}
\State $\hat{\sigma} \leftarrow 0.9\hat{\sigma} + 0.1\norm{\widehat{\g}_t}$ \Comment{update norm estimate}
\end{algorithmic}
\end{algorithm}

\begin{algorithm}[t]
\caption{Momentum-Centric Worker Proof Generation}
\label{alg:momentum-worker}
\begin{algorithmic}[1]
\Require task $(\thetaVec_t, B_t, \bm{m}_t)$
\State Compute gradient $\g \leftarrow \nabla_{\thetaVec}\Lcal_{B_t}(\thetaVec_t)$
\State Compute scalar $a \leftarrow \ip{\g}{\frac{\bm{m}_t}{\norm{\bm{m}_t}}}$ \Comment{project onto momentum}
\State Submit scalar $a$ \Comment{single scalar per worker}
\end{algorithmic}
\end{algorithm}

\section{Experimental Protocol (Overview)}
We defer full reproducibility details (hashing, scripts, adversary injection) to Appendix~\ref{app:repro}. Core ablations:
(i) $K$ sweep (communication vs perplexity), (ii) codebook rank $r$ sweep and captured energy $\gamma_t$, (iii) heterogeneity/straggler scenarios, (iv) adversary fraction vs verification rate $p$, (v) gradient variability analysis across batches, and (vi) momentum-centric vs random direction comparison.

\paragraph{Variant comparison.}
For neural network training (GPT-2 on language modeling), we compare:
\begin{itemize}[leftmargin=*]
  \item \textbf{Full SGD}: Baseline with full gradient synchronization
  \item \textbf{HoloGrad-Random}: $K$ random unit directions, scale-corrected synthesis
  \item \textbf{HoloGrad-ADC}: $K$ subspace directions with Streaming Oja-QR
  \item \textbf{Momentum-Centric HoloGrad}: Single momentum direction, 1 scalar per worker
\end{itemize}
We report training loss, communication cost (scalars per step), and relative efficiency (loss reduction as \% of full SGD).

\section{Limitations}
Key open issues: (i) sample-efficiency at very large $D$ may still require large $K$ without a strong codebook; (ii) practical performance of forward-mode/JVP in mainstream frameworks; (iii) nondeterminism requires careful $\epsilon$ selection; (iv) extending theory to robust aggregation and adversarial settings requires additional assumptions on bounded projections and corruption rates; (v) momentum-centric variant achieves only $\approx 30$--$40\%$ of full SGD efficiency, trading convergence speed for communication reduction; (vi) momentum direction requires a warm-up period (first few steps may use random directions).

\section{Conclusion}
HoloGrad reframes distributed pretraining as a verifiable, mining-style protocol by replacing tensor synchronization with scalar PoGP proofs, enabling permissionless heterogeneous participation. We identify a fundamental challenge for neural networks: gradient directions vary dramatically across mini-batches, rendering random-direction estimators inefficient when $K \ll D$.

To address this, we present two complementary variants:
\begin{itemize}[leftmargin=*]
  \item \textbf{HoloGrad-ADC}: Uses a learned low-rank subspace (Streaming Oja-QR) for settings where gradients occupy a stable subspace.
  \item \textbf{Momentum-Centric HoloGrad}: Projects onto the coordinator's momentum direction, achieving $\approx 30$--$40\%$ of full SGD efficiency with only 1 scalar per worker---a $64\times$ communication reduction compared to $K=64$ random directions.
\end{itemize}

We provide descent guarantees for both variants, analyze gradient subspace variability in neural networks, and define incentive-compatible verification/slashing under an explicit PoGP model. The choice of variant depends on the application: ADC for convex problems or stable-subspace settings, momentum-centric for neural network training under extreme communication constraints.

\bibliographystyle{plain}
\bibliography{refs}

\appendix
\section{Full Reproducibility Protocol}
\label{app:repro}

This appendix provides an end-to-end, ``clone-and-run'' protocol for reproducing the core claims:
(i) training progress under HoloGrad vs baselines, (ii) scaling with number of proofs $K$,
(iii) benefits of the Adaptive Direction Codebook (ADC), and (iv) robustness under heterogeneity
and adversarial submissions with PoGP sampling verification and slashing.

\subsection{Repository Layout (expected)}
Create a repository with the following structure (names are suggestions; adjust as needed):
\begin{itemize}[leftmargin=*]
  \item \texttt{main.tex}, \texttt{refs.bib}
  \item \texttt{repro/}
    \begin{itemize}
      \item \texttt{configs/} (YAML experiment configs)
      \item \texttt{scripts/} (launch scripts)
      \item \texttt{src/} (python modules)
      \item \texttt{logs/} (stdout logs)
      \item \texttt{runs/} (checkpoints, metrics)
    \end{itemize}
\end{itemize}

\subsection{Environment and Dependencies}
\paragraph{Hardware.}
All experiments can be run on a single machine with one GPU for proof-of-concept scaling.
For distributed/heterogeneous tests, multiple machines or containers are recommended.
Minimum recommended single-node baseline:
\begin{itemize}[leftmargin=*]
  \item GPU: 1$\times$ 24GB VRAM (or larger)
  \item CPU: 16+ cores
  \item RAM: 64GB+
  \item Disk: 500GB+
\end{itemize}

\paragraph{Software.}
\begin{itemize}[leftmargin=*]
  \item Ubuntu 22.04 (or similar Linux)
  \item Python 3.10+
  \item CUDA 12.x + matching PyTorch build
  \item Git, \texttt{tmux} (optional), Docker (optional)
\end{itemize}

\paragraph{Python packages.}
Pin exact versions to reduce nondeterminism:
\begin{itemize}[leftmargin=*]
  \item \texttt{torch}, \texttt{transformers}, \texttt{datasets}, \texttt{accelerate}
  \item \texttt{numpy}, \texttt{scipy}
  \item \texttt{pyyaml}, \texttt{tqdm}
  \item \texttt{wandb} (optional)
\end{itemize}

\subsection{Determinism and Hash Commitments}
HoloGrad relies on determinism for (a) directional regeneration from seed and (b) PoGP verification.

\paragraph{Global seed.}
Choose a global seed \texttt{S0}. For every training step $t$ and proof index $j$,
derive a per-proof seed via a cryptographic hash:
\[
s_{t,j} = \mathrm{Hash}(\texttt{S0} \,\|\, H(\theta_t)\,\|\, H(B_t)\,\|\, t \,\|\, j).
\]
Here, $H(\theta_t)$ is a commitment to the checkpoint (e.g., SHA-256 of serialized weights),
and $H(B_t)$ is a commitment to the minibatch identity (dataset name, shard id, sample ids).

\paragraph{Direction generator.}
We fix the full-space generator:
\[
z \sim \mathcal{N}(0,I_D)\ \text{from seeded PRNG},\qquad \vdir = \frac{z}{\|z\|}.
\]
This implies $\sigma^2 = 1/D$ and the scale-corrected synthesis uses $1/\sigma^2 = D$.

\paragraph{ADC codebook commit.}
For ADC, commit to \codebookid\ at each step (or epoch) by hashing the matrix $\U_t$:
\[
H(\U_t)=\mathrm{SHA256}(\mathrm{Serialize}(\U_t)).
\]
Workers use the advertised \codebookid\ to load the exact same $\U_t$.

\subsection{Core Components to Implement}
A minimal implementation needs the following modules:

\paragraph{(1) Coordinator.}
Responsibilities:
\begin{itemize}[leftmargin=*]
  \item Maintain checkpoint $\theta_t$.
  \item Publish tasks $\{s_{t,j}\}_{j=1}^K$ and minibatch id $B_t$.
  \item Collect proofs $(s,a)$.
  \item Aggregate scalars (mean or trimmed mean) into $\{\tilde a_j\}$.
  \item Reconstruct directions $\vdir_j=\mathrm{Dir}(s_{t,j})$.
  \item Synthesize $\widehat g_t = \frac{1}{K}\sum_j \frac{1}{\sigma^2}\tilde a_j \vdir_j$.
  \item Update: $\theta_{t+1}=\theta_t-\eta \widehat g_t$.
  \item (ADC) Update $\U_t$ via Streaming Oja-QR.
\end{itemize}

\paragraph{(2) Worker.}
Responsibilities:
\begin{itemize}[leftmargin=*]
  \item Receive $(\theta_t,B_t,s)$ and optional \codebookid.
  \item Reconstruct $\vdir=\mathrm{Dir}(s)$ (or subspace: $\vdir=\U_t z(s)$).
  \item Compute directional derivative $a=\ip{\nabla \Lcal_{B_t}(\theta_t)}{\vdir}$.
  \item Submit proof $(s,a)$ with optional signature/attestation.
\end{itemize}

\paragraph{(3) Verifier.}
Responsibilities:
\begin{itemize}[leftmargin=*]
  \item Sample each proof with probability $p_{\mathrm{verify}}$.
  \item Recompute $a^{*}$ under identical $(\theta_t,B_t,s)$.
  \item Accept if $|a-a^{*}|\le \epsilon$, else flag and apply slashing.
\end{itemize}

\subsection{Directional Derivative Computation}
Two acceptable approaches:

\paragraph{Option A (reference, simplest): full backprop then dot.}
Compute $\g_B(\theta_t)=\nabla_{\theta}\Lcal_B(\theta_t)$ and set $a=\ip{\g_B}{\vdir}$.
This is correct but may be expensive.

\paragraph{Option B (preferred): forward-mode / JVP.}
Compute $a$ directly as a directional derivative (JVP) without materializing full $\g_B$.
The protocol correctness does not depend on which method is used, but cost and feasibility do.
Benchmark and report both where possible.

\subsection{Datasets and Models}
\paragraph{Recommended minimal setting (fast).}
\begin{itemize}[leftmargin=*]
  \item Dataset: WikiText-103 or OpenWebText subset
  \item Model: GPT-2 small (117M) or similar
  \item Sequence length: 256--512
\end{itemize}

\paragraph{Scaling setting (optional).}
\begin{itemize}[leftmargin=*]
  \item Dataset: C4 subset
  \item Model: 300M--1B parameter decoder-only transformer
  \item Sequence length: 512--1024
\end{itemize}

\subsection{Experiment Matrix}
All experiments log: training loss, validation loss/perplexity, throughput (tokens/s),
verification overhead, rejection/slash counts, and (ADC) captured energy ratio $\gamma_t$.

\paragraph{Baseline comparisons.}
\begin{itemize}[leftmargin=*]
  \item \textbf{BP-AllReduce}: standard distributed training (if available).
  \item \textbf{HoloGrad-Full}: full-space seeded unit directions, scale-corrected synthesis.
  \item \textbf{HoloGrad-ADC}: subspace directions with Streaming Oja-QR codebook.
\end{itemize}

\paragraph{Ablation A (number of proofs).}
Sweep $K\in\{8,16,32,64,128,256\}$ for fixed compute budget.
Report convergence vs wall-clock and vs total proofs processed.

\paragraph{Ablation B (codebook rank).}
For ADC, sweep $r\in\{8,16,32,64,128\}$.
Report $\gamma_t$ and convergence speed.

\paragraph{Ablation C (heterogeneity/stragglers).}
Simulate heterogeneous workers by adding random delays and compute caps.
Compare synchronous collection vs time-windowed collection (collect-first-$K$).

\paragraph{Ablation D (adversaries + robust aggregation).}
Inject adversarial fraction $\alpha\in\{0,0.1,0.2,0.3\}$:
\begin{itemize}[leftmargin=*]
  \item \emph{Random}: submit random $a$ values.
  \item \emph{Sign-flip}: submit $a=-a_{\text{honest}}$.
  \item \emph{Extreme}: submit $a$ scaled by large factor.
\end{itemize}
For each, sweep trimming $\tau$ and report bias/variance tradeoffs.

\paragraph{Ablation E (verification rate).}
Sweep $p_{\mathrm{verify}}\in\{0.0,0.01,0.05,0.1,0.2\}$ and tolerance $\epsilon$.
Report: detection probability, false positive rate (honest flagged), total verification cost.

\subsection{Metrics and Reporting}
\paragraph{Training quality.}
Validation perplexity (or loss) vs steps and vs wall-clock.

\paragraph{Communication.}
Bytes uploaded per worker per step. HoloGrad should be $O(1)$ scalars plus metadata.

\paragraph{Verification overhead.}
\[
\text{Verifier-cost fraction} = \frac{\text{time spent verifying}}{\text{total step time}}.
\]

\paragraph{Security/economics.}
Report empirical $P(\mathrm{caught})$ vs theoretical $1-(1-p)^{qm}$ under controlled injections.

\subsection{Recommended Default Hyperparameters}
\begin{itemize}[leftmargin=*]
  \item Learning rate $\eta$: match baseline optimizer scale; start with $1\mathrm{e}{-4}$ to $3\mathrm{e}{-4}$ for GPT-2-small.
  \item Proof count $K$: 64 for single-GPU PoC.
  \item Verification rate $p_{\mathrm{verify}}$: 0.05.
  \item Verification tolerance $\epsilon$: $1\mathrm{e}{-4}$ (adjust for numeric nondeterminism).
  \item Trim rate $\tau$: 0.1 for moderate adversary rates.
  \item ADC rank $r$: 32.
  \item Oja step size $\alpha_t$: constant $1\mathrm{e}{-3}$ or decay $\alpha_t=\alpha_0/\sqrt{t}$.
  \item QR period $T_{\mathrm{qr}}$: 100 steps.
\end{itemize}

\subsection{Step-by-Step Execution (example)}
\paragraph{1) Prepare environment.}
\begin{itemize}[leftmargin=*]
  \item Install pinned dependencies.
  \item Download dataset shards and write sample-id indices for minibatch commitments.
\end{itemize}

\paragraph{2) Run baseline.}
Train for $T$ steps (e.g., $T=20{,}000$) with standard optimizer and log metrics.

\paragraph{3) Run HoloGrad-Full.}
\begin{itemize}[leftmargin=*]
  \item Fix \texttt{S0}.
  \item For each step, generate $K$ seeds $s_{t,j}$.
  \item Collect $K$ proofs, synthesize $\widehat g_t=\frac{D}{K}\sum_j \tilde a_j\vdir_j$.
\end{itemize}

\paragraph{4) Run HoloGrad-ADC.}
\begin{itemize}[leftmargin=*]
  \item Initialize $\U_0$ with random orthonormal columns.
  \item Use $\vdir=\U_t z$ with fixed $z\sim\mathcal{N}(0,I_r)$.
  \item Update $\U_t$ using Streaming Oja-QR.
\end{itemize}

\paragraph{5) Robustness tests.}
Repeat (3)--(4) with adversary injections, trimming, and PoGP verification.

\subsection{Sanity Checks}
\begin{itemize}[leftmargin=*]
  \item \textbf{Unbiasedness check (full-space):} for fixed $\theta$, estimate
  $\frac{1}{K}\sum_j \frac{1}{\sigma^2}(g\cdot \vdir_j)\vdir_j$ and compare to reference $g$.
  \item \textbf{Scale check:} verify that omitting $1/\sigma^2$ yields a factor mismatch of approximately $1/D$ under unit-norm directions.
  \item \textbf{Verifier consistency:} for honest proofs, measure $|a-a^{*}|$ distribution to set $\epsilon$.
\end{itemize}

\subsection{Artifact Checklist}
To claim reproducibility, include:
\begin{itemize}[leftmargin=*]
  \item Exact commit hash of code and dependency lockfile.
  \item Global seed \texttt{S0}.
  \item Dataset name + exact subset/shard ids + sample-id index files.
  \item Training config YAMLs (all ablations).
  \item Logs and summary tables (CSV) for metrics.
  \item Plots: loss/perplexity curves, verifier overhead, detection rates, $\gamma_t$ curves.
\end{itemize}

\subsection{Notes on Numerical Nondeterminism}
Even with fixed seeds, GPU kernels may be nondeterministic. We recommend:
\begin{itemize}[leftmargin=*]
  \item Using deterministic flags where available.
  \item Logging floating-point settings and hardware.
  \item Setting $\epsilon$ using empirical quantiles of $|a-a^{*}|$ for honest runs.
\end{itemize}
\end{document}
